%!TEX root = ../lections.tex
\subsection{Волны в линиях передачи}
Основное - передача электромагнитной волны от источника к приёмнику.
Основная теория - уравнения Максвелла. Задача сволится к краевой задаче.

 Для монохроматических полей(синусоидально завясящих от времени):
  \begin{equation}
	\vec{E} = \vec{E}_0(\vec{r})\cdot \exp(i\omega t)
\end{equation}
где $\vec{E}_0(\vec{r})$ - комплексная амплитуда.
Зависимость от времени присутствует в виде отдельного слан=гаемого - это приемущество, помогает превратить уравнения Максвелла в алгебраическое выражение за счёт разделения переменных.
 \begin{equation}
 	\vec{E}_{nast} = Re(\vec{E}_0(\vec(r))\cdot \exp(e\omega t))
\end{equation}
\begin{equation}
	E_x = |E_x(\vec(r)| \exp(i\omega t)
\end{equation}
Чаще всего будем писать так : $\vec{E}_0\qty(\vec{r}), \vec{H}_0\qty(\vec{r})$ - комплексные аплитуды(чтобы перейти к реальным нужно домножить на $\exp(i\omega t)$  и взять реальную часть)
 
 	Уравнения Максвелла
\begin{gather}
 	rot(\vec{H}(\vec(r)) = \frac{4\pi}{c} \cdot \vec(j)^e + \frac{i \omega}{c} \cdot \epsilon \vec{E}
\\
 	rot(\vec{E}) = -\frac{i \omega \mu}{c} \vec{H}
\\
 	\Div(\epsilon \vec{E}) + \\frac{4\pi}{(i\omega)} \cdot(\vec(j)^e) = 0
\\
 	\Div(\mu \vec{H}) = 0
\\
 	\vec{B} = \mu \vec{H} = rot(\vec{A}_e)
\\
 	\vec{E} = - \nabla\phi - (1/c) \frac{\partial \vec{A}^e}{\partial t} = - \nabla{\varphi} - \frac{i \omega}{c} \vec{A}^e
\end{gather}
$\pdv{t}$ заменяется на  $i \omega $ где 
$\phi$ - сказялрный потенциал, $\vec{A}^e$  - векторный потенциал;
Одно можно выразиь через другое из условия нормировки Лоренца:
\begin{gather}
	\Div{\vec{A}} + \frac{\epsilon \mu}{c} i \omega \Phi = 0\\
	\phi = - \frac{c}{i \omega \epsilon \mu} \Div{\vec{A}}\\
	\vec{H} = \frac{1}{\mu} rot{\vec{A}}\\
	\vec{E} = \frac{1}{i k \epsilon \omega} (\nabla\\Div + k^2)\\
 \vec{A}^e
\end{gather}
$ \epsilon \mu $  во всём курсе будут рассматриваться, как не завясящие от времени и координат
Линия передачи -  ...
Существуют два направления: продольное и поперечное.
При описании через векторный потенциал проще перейти к уравнению Гемгольца:
\begin{equation}
	\delta \vec{A}_e + K^2 \vec{A}_e = 0
\end{equation}

Сначала рассмотри без источников( поэтому в формле справа 0) свободные волны.
Как записывется поле для электромагнитной волны в линии передачЖ
\begin{equation}
	\vec{E}(\vec{r}_\perp , z , t) = \vec{E}_0(\vec{r}_\perp) \exp{i\omega - h z)}
\end{equation}

 h- продолное волновое число( постоянная распространения)

 Мы рассмотрим волну бегущую по оси z.
 \begin{equation}
 	E_{x}{nast} = Re{E_x}= |E_x(\vec{r}_\perp)| cos(\omega t - hz + \phi(\vec{r}_\perp))
 \end{equation}

 Самый простой метод - нужно задать векторный потенциал в виде:
 \begin{equation}
 	\vec{A}_e = \phi^e(\vec{r}_\perp) \exp(-i h z) \vec{z}_0
 \end{equation}

Мы его задаём проще: направляем по оси z, зависимость такая же, как и у поля, а как зависит от $\vec{r}_\perp)$  не знаем.
	$\phi^e(\vec{r}_\perp)$	поперечная волновая функция.
Покажем, что решение уравения Максвелла можно найти в таком виде:
\begin{gather}
	\Div(\vec{A}_e)= -i h \phi^e(\vec{r}_\perp) \exp(-i h z)\\
	\nabla(\Div(\vec{A}_e)) = (- h^2 \phi^e(\vec{r}_\perp) \vec{z}_0 - i h \nabla_\perp \phi^e(\vec{r}_\perp ) \exp(-i h z)\\
	rot(\vec{A}_e)) = [\nabla(A^e_z,\vec{z}_0 )] = [\nabla_\perp \phi^e(\vec{r}_\perp, \vec{z}_0 ]\\
	\nabla = \nabla_\perp + \vec{z}_0 \pdv{z}\\
\end{gather} 

TM - волна- поперечная магнитная волна( магнитное чисто поперечное, а Е и продолное и поперечное)
\begin{gather}
	E_z = \frac{\kappa^2}{i k_0 \epsilon \mu} \nabla_\perp \phi^e(\vec{r}_\perp \exp{i\omega - h z)}\\
	\vec{E}_\perp = - \frac{h}{k_0 \varepsilon \mu} \nabla_\perp \phi^e(\vec{r}_{\perp} \exp{i\omega - h z)} \\
%
	\vec{H}_\perp = \frac{1}{\mu} [\nabla_\perp \phi^e(\vec{r}_\perp, \vec{z}_0] \exp{i\omega - h z)}\\
	H_z = 0
\end{gather}
\begin{gather}
	\vec{E} = \vec{E}_\parallel + \vec{E}_\perp\\
	\vec{E}_\parallel = \vec{z}_0 E_z
\end{gather}
\begin{equation}
	\vec{A} + \frac{\epsilon \mu}{c^2} \frac{\partial^2\vec{A}_e}{\partial{t}^2} = 0
\end{equation}
Это уравнение Гемгольца при произвольной зависимости от времени.
\begin{gather}
	\nabla(A_z) + k^2 A_z = 0\\
	K^2 = \frac{\omega}{c^2} \varepsilon \\
	\nabla =  \pdv[2]{x} + \pdv[2]{y}+ \pdv[2]{z}\\
	\nabla_\perp =  \pdv[2]{x} + \pdv[2]{y}
\end{gather}
Это уравнение Гемгольца при гармонической зависимости от времени.
\begin{equation}
	\frac{\partial^2}{\partial{z}^2} = -h^2
\end{equation}
\begin{gather}
	A_z^e = \phi^e(\vec{r}_{\perp} \exp{i \omega - h z)}\\
	\delta_\perp + (k^2 - h^2) \phi^e = 0
\end{gather}
Функция $\phi^e$ должна удовлетворять двумерному уравнению Гемгольца.
\begin{equation}
	\kappa^2 = k^2 - h^2 = K_0^2 \varepsilon \mu - h^2\\
	\delta_\perp \phi^e + \kappa^2 \phi^e = 0
\end{equation}
где $\kappa^2$ - поперечное волновое число.
Если $\phi^e$ удовлетворяет уравнению выше, то и удовлетворяет уравнению Максвелла.

TЕ - волна - поперечная элекрическая волна.

Уравнения Максвелла симметричны относительно полей, но мы получили неравноправие векторов. Это просто одно из решений. Но есть ещё	одно решение. И чтобы его получиь воспользуемся принципом двойственности.

TЕ - волна - поперечная элекрическая волна.
\begin{gather}
%
H_z = \frac{\kappa^2}{i k_0 \epsilon \mu} \nabla_\perp \phi^m(\vec{r}_{\perp} \exp{i\omega - h z)}\\
%
\vec{H}_\perp = - \frac{h}{k_0 \varepsilon \mu} \nabla_\perp \phi^m(\vec{r}_{\perp} \exp{i\omega - h z)} \\
%
\vec{E}_\perp = - \frac{1}{\epsilon} [\nabla_\perp \phi^m(\vec{r}_{\perp}, \vec{z}_0] \exp{i\omega - h z)}\\
E_z = 0
%
\end{gather}
$\phi$ не обязано быть таким же, пожтому мы ставим индексы $m$ и $е$, но должно удовлетворять уравнению:
\begin{equation}
	\delta_\perp \phi^m + \kappa^2 \phi^m = 0
\end{equation}
ТЕМ - волна.
Когда $\kappa = 0, h = k$ тогда $H_z = E_z = 0$. Это чисто поперечная волна(ТЕМ)
\begin{gather}
	H_z = E_z = 0\\
	\vec{E}_\perp = - \frac{1}{\sqrt{\varepsilon \mu}} \nabla_\perp \phi(\vec{r}_\perp \exp{i\omega - h z)} \\
	\vec{H}_\perp = \frac{1}{\mu} [\nabla_\perp \phi(\vec{r}_\perp), \vec{z}_0] \\\exp{i\omega - h z)}
\end{gather}
Эта волна удовлетворяет уравнению:
\begin{equation}
	\nabla_\perp \phi = 0
\end{equation}
Свойства уравнений (не знаю каких)
\begin{enumerate}
	\item Поля выражены через скалярные функции
	\item Продольные компоненты пропорциональны этой волновой функции
	\item Поперечные компоненты пропорциональныградиенту этой волновой функции
	\item  Нужно задать $\phi^e и \phi^m$ , чтобы можно было полностью описать поля.
\end{enumerate}