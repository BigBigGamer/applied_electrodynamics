\subsection{Другие подходы к описанию волн в линии передачи}
Сначала замечание.
%
Мы искали решение в виде гармонической волны $\exp{i(\omega t - h z)}$. Можно было бы подойти издали, и искать решение в виде $\vec{E} \sim f(z) \vec{F(\vec{r_\perp})} \exp{i \omega t}$: , то есть делать методом разделения переменных.
%
Из уравнениния Максвелла нетрудно получить уравнение для функции $f$:
\begin{equation}
	\dv[2]{f}{z} + h^2 f = 0
\end{equation}
\begin{equation}
	f \sim \exp{\pmi h z}
\end{equation}
$h$ - продольное волновое число.
%
А как выглядит решение если $h = 0, \kappa = k, \omega = \omega_{cr}$ ?
%
По нашему старому подходу поле просто константа.
%картинка
Но из этого подхода получается другое:
%картинка
Это какие-то линейные функции.
\begin{equation}
	f = C_1 + C_2 z
\end{equation}
$C_1$ и $C_2$ зависят от граничных условий.
%
Почему же результаты разные? Строго говоря, мы ничего нового не получили, просто нужно по-другому посмотрели новым подходом.
\begin{equation}
	E_{x} \sim{C_1 \exp{+i h z} + C_2 \exp{-i h z}} = A_1 +A_2 z
\end{equation}
Случай $\alpha \rightarrow 0$: очевидно, что существует набор констант. !!А что, собственно, за альфа?!!
%А что, собственно, за альфа?
Чтобы получилась линейная функция: $C_1 \rightarrow \infty, C_2 \rightarrow \infty$
%
Мы рассматриваем регулярную линию передачи с одним сечением везде. Но если волновод меняет свою площадь сечения?
%
Если сужение плавное, то можем описать тем же подходом, но $h = h(z)$. Нужно решать:
\begin{equation}
	f'' + h^2(z) f = 0
\end{equation}
Есть другой подход, который называется концепция Бриллуена. 
%
Форумулировка: любую моду в линии передачи можно представить в виде суперпозиции плоских однородных волн. 
%
У нас есть линия передачи.
\begin{equation}
	E_{mn} = \Sigma_{\vec{k}} \vec{E_{\vec{k}}} \exp{i(\omega t -\vec{k} \vec{r} )}
\end{equation}
$\vec{k}$- волновой вектор плоской волны. 
\begin{gather}
	(\vec{k})^2 = k_{\perp}^2 + k_z^2\\
	\vec{k} = \vec{z_0} h + \vec{k_{\perp}}\\
	|\vec{k_{\perp}}| = \kappa\\
	k_{z} = h
\end{gather}
$h = k_{z}$- для всех видов волн при любом . У нас получился конус волновых векторов, при том что длины этих волн равны при одной частоте. Различаются только направления.
\begin{equation}
	(\vec{k})^2 ={\frac{\omega}{c}}^2 \epsilon \mu
\end{equation}
%
$h$ и $\kappa$- продольная и поперечная компонента волнового вектора в плоской волне.
%
По сути, то что мы сделали - это разложение в ряд Фурье.
\begin{equation}
	k_z = h, \kappa = k_{\perp}
\end{equation}
Импедансное соотношение тоже имеет другой смысл.
%
Согласно нашей классификации, $E$ перпендикулярно направление распространения(координате $z$).
TM:
\begin{equation}
	\eta_{\perp v} = \frac{E_{\perp}}{H_{\perp}} = \frac{E_{0} \cos{\alpha}}{H_{0}} = \sqrt{\mu}{\epsilon} \frac{h}{k}
\end{equation}
TE:
\begin{equation}
	\eta_{\perp v} = \frac{E_{\perp}}{H_{\perp}} = \frac{E_{0}}{H_{0} \cos{\alpha}} = \sqrt{\mu}{\epsilon} \frac{k}{h}
\end{equation}
где $E_{0}, H_{0}$ истинные длины векторов. 
%
Эти импедансные соотношения становятся понятны. Поперечный волновой импеданс - это отношение проекций поля, спроецированных на поперечную плоскость.
%
Пример. Прямоугольный волновод.
%Картинка волновода
Любое распределение можно представить таким образом:
%картинка где луч отражается от стенки
На картинке волна бежит, отражается и т.д. Так она распространяется.
%
С такой точки зрения понятно создание критического режима.
%
Это случай когда $h = 0, k = \kappa$
%
При этом появится структура стоячей волны, но поток энергии равен 0. 
%
Если хотим $\omega < \omega_{cr}$, волна сможет только экспоненциально затухать. 
%
Существование критического режима приводит к существованию предельного угла.
\subsection{Описание ТЕМ волн в линиях передачи на основе телеграфных уравнений}
Мы знаем, что ТЕМ волна существует только когда число проводников в системе не меньше двух.
%
Рассмотрим двупроводниковую линию:
%
Такие линии передачи имеют прямую связь с теорие LC - цепочек.
%
На этом и базируются телеграфные уравнения.
Можно рассмотреть когда $\lambda >> l$ и ввести погонные параметры: С - погонная ёмкость и L - погонная индуктивность. Это нужно для того, чтобы рассмотреть всю систему как сплошную линию. 
%
В ТЕМ - волне в каждом сечении можно говорить о токе $I$ и зарядах $-q, q$, приходящихся на единицу длины. Ведем ещё одну существенную величину - напряжение:
\begin{equation}
	V = \int_1^2 E_l dl
\end{equation}

И 1 и 2 лежат в одном поперченом сечении. А какой может быть ворма контура? Любую, она зависит только от поперечного сечения $z$.Все величины($V, Q, I$) зависят $z$ от и от $t$
%
Вид с торца:
%
В ТЕМ волне поля:
\begin{equation}
	\vec{E} = \vec{E_\perp}, \vec{H} = \vec{H\perp}
\end{equation}
%
Они потенциальны:
\begin{equation}
	\vec{E_\perp} = - \grad_\perp{\Phi}
\end{equation}
%
И удовлетворяют уравнению:
\begin{equation}
	\grad_\perp{\Phi} = 0
\end{equation}
А так же граничным условиям
\begin{equation}
	\rot\vec{E} \neq 0, \rot\vec{E} = \rot{\grad_\perp{\Phi(\vec(r_\perp)\exp{-i h z})}}
\end{equation}
Теперь получим уравнения, которые свзявают все эти величины.
%
Первый закон Кирхгофа - закон сохранения заряда.
%
Определяем $\Delta{q}$ как величину элементарного заряда приходящуюся на $\Delta{z}$:
\begin{equation}
	\Delta{q} = Q \Delta{z}
\end{equation}
%
Надём приращение за $\Delta{t}$:
\begin{gather}
	t \rightarrow t + \Delta{t}\\
	\Delta{\Delta{q}} = \Delta{q(t + \Delta{t})} - \Delta{q(t)}\\
	\Delta{\Delta{q}} = \Delta{t} [I(z) - I(z + \Delta{z})]
\end{gather}
%
Перейдём к пределу:
\begin{gather}
	\lim_{\Delta{z}) \rightarrow 0, \Delta{z}) \rightarrow 0} \frac{1}{\Delta{z} \Delta{z}}\\
	\pdv{\Delta{q}}{t} = I(z) - I(z + \Delta{z})\\
\end{gather}
И получаем:
\begin{equation}
	\pdv{Q}{t} = -\pdv{I}{z}
\end{equation}
%
это уравнение непрерывности для заряда.
%
Используем свзяь: $Q = CV$
\begin{equation}
	\pdv{I}{z} = - C \pdv{V}{t}
\end{equation}
%
- это первое телеграфное уравнение. 
%
Используем Второй закон Кирхгофа:
\begin{equation}
	\oint_{L} E_l dl = - \frac{1}{c} \pdv{\Delta{\Psi}}{t} = V(z + \Delta{z}) - V(z)
\end{equation}
$\Delta{\Psi} = \frac{$\Delta{L} I}{C} $где поток индукции(или всё же энергии???) через площадку натянутую на контур, а  $\Delta L = L \Delta{z} $- коэффициен самоиндукции данного отрезка(погонная самоиндукция).
%
Мы получаем такое уравнение:
\begin{equation}
	\pdv{V}{z} = - \frac{1}{C^2} L \dv{I}{t}
\end{equation}
- это второе телеграфное уравнение.
%
Эти два уравнения эквивалентны уравнениям для поля. Они завися только от продолных координат и от времени.
%
Можно провести аналогию:
\begin{equation}
	I \Leftrightarrow \vec{H}, V \Leftrightarrow \vec{E}
\end{equation}
Продиффиренцируем по $z$:

- это волновое уравнение.

Если нет дисперсии( и независят от ), извесино решение:

Таким же образом можно оплучить уравнение для тока и его решение.
%
Среда предпологалась однородной, из равества выше следует:

Мы будет интересоваться решением для монохроматической волны:

ТЕМ:

Если песчитать эти уравнения для гармонических процессов:

то мы сразу получим связь между напряжением и током в бегущей волне:

Тогда из телеграфных уравнений получим соотноешение для волнового сопротивления(импеданса):

Здесь более оправдано такое определение:
%
Волновым сопротивлением в линии передачи В ТЕРМИНАХ тока и напряжения называется отношение напряжения к току в бегущей волне.
%
Ранее мы задавали такое же понятие, но  в терминах полей.
%
Данное соотношение полезно при решение задач на отражение. 
%
\subsection{Отражение волн в линии передачи на языке телеграфных уравнений}
%
Пусть есть линия передачи и она соединяет источники электромагнитных колебаний с нагрузкой(потребителем) - любым элементом, характеризующийся напряжением на нём и током, проходящим через него.

Нагрузка - отношение напряжения на нагрузке к току, проходящим через неё.(А вообще она комплексная, так как напряжение и ток - комплексные амплитуды).
%
Пусть там есть источник переменного напряжения.
%
Введём ось  и поместим начало отсчета там, где нагрузка. 
%
Коэффициент отражения  - это отношение напряжения отраженной волны к напржению падающей волны.



- напряжение и сила тока в падающей волне.
%
Если источник монохроматический, то это сумма в падающей и отраженной волне(Не совмем помню, к чему это).
%
Введём
- импеданс в данном поперечном сечении с координатой . 

Посчитаем, чему он равен:

Учтем, что

Теперь сделаем определённые соотношения:

Формула пересчета импедансов:
%
Положим (на входе) и используем формулу Эйлера():

Поставим  и используем формулу для Г:

Она позволяет зная импеданс на нагрузке найти импеданс на источнике( на входе). Это важно для выбора режима работы генератора.
%
Здесь  - длина линии.
%
Эта формула так же позволяет делат пересчет для нескольких линий передачи.
%
Нужно идти справа налево.
%
Согласование линии передачи с нагрузкой, при том, что в линию не отражается от нагрузки( если ). Это значит вся энергия потреблена и не зря использована. 

- условие согласования.
%
\subsection{Волны в диэлектрических линиях передачи}
%
Мы рассматривали металлический волновод. Но сейчас важную роль играют диэлектрические линии передачи. Они хороши тем, что тангенс угла потерь будет мал. А металлические линии могут быть использованы только для простых приборов, но не для связи. А для передачи на много километров используются волоконные линии передачи. Мы рассмотрим волны вдоль диэлектрического слоя(пластинки). Пусть имеется плоскопараллельная пластинка, на ней слой диэлектрика, а в окружающем пространстве . 

Раньше мы не интересовались полями внутри (область1) , но теперь нам нужно нужно такое решение, чтобы сшить наружнюю(область 2) и внутреннюю части. Будем считать, что  одинаково в обеих областях, но разные. Делаем так, потому что требуем, чтобы граничные условия выполнялись при одинаковых , то есть при одинаковых . Будем описывать всё функцией . Она удовлетворяет уравнению Гемгольца:


Из наших соотношений про и :

Займёмся решениями. Нужно ограничение: волна локализованная( с убыванием). Положим , тогда решения ,  действительно и положительно.
%
Значит волна убывает при приближении и удалении от слоя.







