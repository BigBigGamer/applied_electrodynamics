%!TEX root = ../lections.tex
\subsection{О других линиях передачи}

ТЕМ возможно если число проводников больше одного. Мы рассматриваем только зактрытые волноводы.
%
Для передачи ТЕМ волн используют открытые линии.Например двухпроводные линии.
%
Например полосковая линия(две параллельные полоска)
На картинке поперечное сечение.
%
Размер зазора $d<< a$, а - длина линии. Фактически это плоский конденсатор.Основное поле заперто между полосками.
%
Если $d$ - мало, то поле между полосками однородно.
%
Такие линии передачи просто делать: берется один плоский лист, на него наносится диэлектрик, а на диэлектрик другой проводящий лист. И потом их просто режут.
%
Обычно вводится диэлектрик, чтобы можно было управлять свойствами поля.
%
Замечание
%
Без диэлектрика возможно создание ТЕМ - волны. Если заполнили диэлектриком, то и тогда возможно создание ТЕМ - волны.  
%
Если же диэлектрик только внутри, то поле немного исказится. Но ТЕМ волна там невозможна(и ТМ и ТЕ тоже), так как заполнение поперечно-неоднородное.
%
В таком волноводе существуют так называемые гибридные волны (волны у которых $E_z \neq 0, H_z \neq 0$). Это супервозиция ТЕ и ТМ волн.
%
Когда $d<<a$, то волна тоже ТЕМ, хотя она и гибридная, в ней есть продольные компоненты полей(но они малы).
%
П - образный волновод и Н - образный волновод.
%
В них можно создать ТЕМ - волны.
%
Зазор (горизонтали) назывется ёмкостным, а магнитное поле находится в индуктивностях (в вертикалях).
%
Если зазор $d <<a$, о может получится что $a << \lambda$ где $\lambda$ - длина волны. Здесь $\lambda_{cr} \rightarrow 0$ при $d \rightarrow 0$, в то время как для волноводов с прямоугольным и круглым сечениями  $\lambda_{cr} \sim a$.
%
При $d \rightarrow 0$ их можно описать по аналогии со статиткой.
%
\subsection{Замечание о поверхостных зарядах и токах в стенках линии передачи.}
%
Если круглый волновод и низшая мода.
%
Здесь на стенках есть поверхостный заряд:
\begin{equation}
	\rho_{pov} = \frac{E_n}{4 \pi ??}
\end{equation}
Это есть следствие граничных условий.
Так же есть поверхостный ток:
\begin{equation}
	\vec{i_{pov}} = \vec{j_{pov}} = \frac{c}{4 \pi} [\vec{H}, \vec{n}]
\end{equation}
Где $\vec{n}$ - внешняя нормаль (направленна в металл).
%
По внутренней поверхности волновода течет ток. Это важно значит, если мы хотим измерить поле в волноводе, но нужно это делать так, чтобы как можно меньше исказить поле.
%
Щель прорезается вдоль линии тока. Так мы меньше всего исказим поле(Не нарушаяя линии тока).
%
Если наоборот, мы хотим сильно исказить поле, то нужно прорезать щель перепендикулярно протеканию тока.
%
Здесь сказано не про сторонние зараяды и токи, а про присущие самой волне.
%
\subsection{Затухание волн в линиях передачи обусловленное потерями энергии.}
До этого мы считали проводник идеальным: $\sigma = \infty, E_{\tau} = 0, H_n = 0$.
%
В реальных линиях волна распостраняется с затуханием. Потери могут быть связаны с выделением тепла, со средой, в которой распространяется волна и т.п.
%
\begin{enumerate}
	\item Затухание, обусловленное потерями энергии в среде.
	%
	Здесь $\epsilon$ - комплексное, так так есть потери. В любой реальной среде есть потери. Мнимая часть обусловленна проводимостью вещества.
	\begin{equation}
		\epsilon_{k} = \epsilon' + i \epsilon''
	\end{equation}
	Сейчас предпологаем $\sigma_{stenki} = \infty$ , но среда проводящая.
	\begin{gather}
		\epsilon'' = - \frac{4 \pi \sigma}{\omega}\\
		\epsilon = \epsilon' - i \frac{4 \pi \sigma}{\omega}
	\end{gather}
	Есть потери - есть затухание. '' - '' в мнимой части связан с тем, что в волне выбрали $\exp{ i \omega t}$ , а если бы выбрали $\exp{- i \omega t}$ то ,ыл бы '' + ''.
	%
	Дисперсионное уравнение:
	\begin{equation}
		h^2= \frac{\omega^2}{c^2}{\varepsilon \mu} -\kappa
	\end{equation}
	То, что решали до этого подходит и сейчас, если $\varepsilon, \mu$ однородны, при чем не важно комплексные или действительные.
	%
	Также верно дисперсионное уравнение и выражение для $\kappa$, при этом $\kappa$ - действительная величина, определяемая геометрией задачи. Так как $\epsilon$ не действительная, то и $h$   будет комплексной(не смотря даже на то, если $\omega > \omega_{cr}$ )
	Пусть $\mu = 1, \epsilon = \epsilon' + i \epsilon''$
	%
	Тогда 
	\begin{gather}
		h = h' + h''\\
		h'^2 - h''^2 + 2 i h'' h' = \frac{\omega^2}{c^2} \epsilon' - \kappa^2 + i \frac{\omega^2}{c^2} \epsilon''
	\end{gather}
	\begin{gather}
		h' - h'' = \frac{\omega^2}{c^2} \epsilon' - \kappa^2\\
		2 i h'' h' = i \frac{\omega^2}{c^2} \epsilon''
	\end{gather}
	Обычно интересует, когда потери малы. Если потери велики, то волна затухает.
	%
	Предположим, что потери малы: $\epsilon''  \rightarrow 0 $.
	
	%
	Тогда предположительно:$h''  \rightarrow 0 $
	\begin{gather}
		(h')^2 = \frac{\omega^2}{c^2} \epsilon' - \kappa^2\\
		h' = \sqrt{\frac{\omega^2}{c^2} \epsilon' - \kappa^2}\\
		h'' = \frac{\omega^2}{c^2} \frac{\epsilon''}{2 \sqrt{\frac{\omega^2}{c^2} \epsilon' - \kappa^2}}
	\end{gather}
	Обычно интересует когда потери малы.
	В этом приближении реальная часть не отличается от того, что мы рассматривали раньше, а мнимую определяем сейчас.
	%
	Так можно делать только когда
	\begin{equation}
		h'' << h' = \sqrt{\frac{\omega^2}{c^2} \epsilon' - \kappa^2}
	\end{equation}
	Если $\omega$ близко к $\omega_{cr}$, то затухание возрастает и формулы не годятся.
	%
	Когда $\omega$ далеко от $\omega_{cr}$,то можем так писать.
	\begin{equation}
		h'' = \frac{\epsilon'' k_{0/^2}}{2 h'}
	\end{equation}
	Всё определяется $\epsilon''$. У диэлектриков, которые используют, измерют $\epsilon''$ и $\epsilon'$ .
	%
	И вводят тангенс угла потерь:
	\begin{equation}
		\tg{\delta} = \frac{|\varepsilon''|}{\epsilon'}
	\end{equation}

	Вещество плохо поглощает, если: $\tg{\delta} << 1$
	%
	$\epsilon''$ зависит от частоты.
	%
	В радиодиапазоне и СВЧ $\lambda \sim$ см, дм и м, то $\tg{\delta} \sim 10^{-2}, 10^{-3}$
	%
	На оптических частотах $\tg{\delta} \sim 10^{-8}$ 
	%
	Поля
	\begin{equation}
		\vec{E}, \vec{H} \sim \exp{i(\omega t - h z)} = \exp{i(\omega t - h' z)} \exp{h'' z}
	\end{equation}
	$\exp{h'' z}$описывает затухание т.к.  $h'' \sim \epsilon'' < 0$
	%
	Тогда если волна бежит в $+z$ направлении, то 
	\begin{equation}
		\vec{E}, \vec{H} \sim \exp{-|h''| z}
	\end{equation}
	Нарисуем моментальный снимок поля. Здесь $\omega > \omega_{cr}$.
	%
	Снимок
	%
	$\exp$не сдвигается(сдвигается только то, что внутри неё.)
	%
	Характерная длина затухания : $L = \frac{1}{h''}$ - aмплитуда уменьшается в $e$ раз.
	%
	$h''$- описывет длину затухания.
	\item{Затухание, обусловленное потерями в стенках волновда}
	Рассмотрим линию передачи произвольной формы поперечного сечения. Она заполнена либо действительными $\epsilon, \mu$ , либо ничем не заполнена. Также мы рассматриваем хороший проводник, у которого $4 \pi \sigma >> \omega$.То есть $\epsilon'' >> \epsilon'$ для стенки волновода.
	\begin{gather}
		\epsilon_{stenki} = \epsilon' + i \epsilon''\\
		\epsilon'' = - \frac{4 \pi \sigma}{\omega}\\
		|\epsilon''| >> \epsilon'
	\end{gather}
	На хорошем проводнике выполняется граничное условие Леонтовича.
	%
	В проводнике волна быстро затухает:
	\begin{equation}
		E_x \sim \exp{- \frac{1+i}{\delta} x} \exp{ i \omega t}
	\end{equation}
	$\delta = \frac{c}{\sqrt{2 \pi \sigma \mu \omega}}$- толщина скин слоя.
	%
	При этом $k_{ctenki} = \frac{1+i}{\delta}$
	%
	Поля на границе удослетворяют условию:
	\begin{equation}
		\vec{E_{\tau}} = \eta_s [\vec{H_{\tau}}, \vec{n}]
	\end{equation}
	$\vec{n}$где  - нормаль вглубь проводника.
	%
	Имеем дело с полем с фиксированной структурой(оно справо от границы), в проводнике). Это позволяет написать граничное условие слева от границы.
	При этом
	\begin{equation}
		\eta_s = \sqrt{\frac{\mu_{stenki}}{\epsilon_{stenki}}} = \sqrt{i \frac{\mu \omega}{4 \pi \sigma}}
	\end{equation}
	По сути здесь решение точное для металлического полупространства.
	%
	У нас же стенка конечной толщины. Но мы можем применить это граничное условие.
	%
	Но граница у нас кривая. В каждой точке структура поля своя. Если искривление мало, то можем малые плоские участки рассматривать отдельно.
	%
	Также толщина стенки у нас конечна. Но если $d >> \delta$, то можно применить это условие, т.к. поле успеет затухнуть не дойдя до второй границы.
	%
	Таким образом, мы используем три условия:$1) \sigma >> \omega, 2) \delta << R_{kr}, 3) \delta << d_{stenki}$
	%
	Тогда мы имеем право воспользоваться граничным условием Леотовича.
	%
	Задача: рассчитать 
	%
	Для расчета можно воспользоваться энергетическим методом. Используем закон сохранения энергии.
	%
	Рассмотрим два сечения $z$ и $z+ \Delta{z}$
	%
	Пусть П - поток энергии через поперечное сечение волновода. 
	%
	П(z) - входящий, П($z+ \Delta{z}$)  - выходящий.
	%
	$\Pi(z+ \Delta{z}) < \Pi(z)$ так как часть энергии ушла в стенку (из-за наличия потерь).
	\begin{equation}
		\Pi(z) = \Pi(z+ \Delta{z}) + \Delta{P_{stenki}}
	\end{equation}
	\begin{equation}
		\Delta{P_{st}} = P_{st} \Delta{z}
	\end{equation}
	$P_{st}$- количество энергии уходящей в стенку, в расчете на единицу длины. То есть $P_{st}$ - погонная мощность потерь.
	%
	Здесь и далее речь идёт о среднем потоке энергии !!Выделить жирным!!
	\begin{equation}
		P_{st} = \frac{\Pi(z+ \Delta{z}) - \Pi(z)}{\Delta{z}}, \Delta{z} \rightarrow 0
	\end{equation}
	Далее дифференциальный закон сохранения энергии(по сути):
	\begin{equation}
		\dv{\Pi}{z} = - P_{st}
	\end{equation}
	\begin{gather}
		\vec{E}, \vec{H} \sim \exp{h'' z}\\
		|\vec{E}|, |\vec{H}| \sim \exp{h'' z}\\
		\Pi \sim |\vec{E}| |\vec{H}| \sim \exp{2 h'' z}
	\end{gather}
	П - полный поток энергии.
	\begin{equation}
		\dv{\Pi}{z} = 2 h'' \Pi
	\end{equation}
	\begin{equation}
		h'' = - \frac{P_{st}}{ 2 \Pi}
	\end{equation}
	\begin{equation}
		\Delta{P_{st}} = \iint{\overline{S_n}} dS = \iint_{\Delta{S_{bok}}}{\frac{c}{8 \pi} Re[\vec{E}, \vec{H^*}]\vert_n} dS
	\end{equation}
	Используя граничное условие Леонтовича, перепишем:
	\begin{equation}
		\Delta{P_{st}} = \frac{c}{8 \pi} \oint_L \Re{\eta_{pov}} |\vec{H_{\tau}}|^2 dl \Delta{z}
	\end{equation}
	При этом:
	\begin{gather}
		\eta_{s} = \eta_{pov} = \sqrt{\mu_{st}}{\epsilon_{st}} = \sqrt{i \omega \mu}{4 \pi \sigma_{stenki}}\\
		\mu_{st} = \mu\\
		|\eta_{s}| << 1
	\end{gather}
	$\eta_{pov}$ - поверхостный ...
	\begin{equation}
		h'' = \frac{P_{st}}{ 2 \Pi} = 
		- \frac{\Re{\eta_{s}} \oint_L{|\vec{H_{\tau}}|^2} dl}{2 \Re{\eta_{\perp v}} \iint_\Sigma |\vec{H_{\perp}}|^2dS}
	\end{equation}
	Но сами поля нам по сути неизвестны.
	\begin{gather}
		\vec{H_{\tau}} = \vec{H_{\tau}}^{0}\\
		\vec{H_{\perp}} = \vec{H_{\perp}}^0\\
		\eta_{\perp v} = \eta_{\perp v}^0
	\end{gather}
	\begin{equation}
		|h''| = \frac{1}{L_{zat}}
	\end{equation}
	$L_{zat}$ - расстояние, на котором амплитуда убывает в е раз.
	 %
	Сделаем качественную оценку:
	\begin{gather}
		\eta_{s} \sim \sqrt{\omega}{\sigma}\\
		Re{\eta_{\perp v}} \sim 1\\
		\oint_L{|\vec{H_{\tau}}|^2} dl \sim 
			\overline{|\vec{H_{\tau}}|^2}^{\,perimetr} l\\
		\iint_\Sigma |\vec{H_{\perp}}|^2 dS \sim \overline{|\vec{H_{\perp}}|^2}^{\, \Sigma} \Sigma\\
		|h''| \sim {\sqrt{\omega}{\sigma} \frac{l}{\sigma}} \sim{\sqrt{\omega}{\sigma} \frac{1}{l}}
	\end{gather}
	Нарисуем график:
	 %
	 Формула справедлива при $\omega >> \omega_{cr}$  и растет как $\sqrt{\omega}{\sigma}$.
	 %
	 Приближаясь к $\omega_{cr}$ наше приближение неправильное.
	 %
	 Легко увидеть, что $h''$ сильно возрастает при приближении к 
	 %
	 Между этими участками мы оценку не делаем, но знаем, какой там вид у графика.
	 %
	 У этого графика есть минимум, так находится $\omega_{opt}$ - наиболее оптимальная частота для передачи сигнала. Но существует такой тип волны и такая линия передачи, что такой рост вверх (при $\omega >> \omega_{cr}$) отсутствует. Это волна типа $TE_{01}$ в круглом волноводе (хотя, если обобщить $TE_{0 n}$).
	 %
	 Такая волна симметричная. Здесь в центре поле Е 0 и на стенках поле Е тоже 0.А магнитное поле имеет на стенке только продольную компоненту:

	 Картинка сбоку:

	 При $\omega  \rightarrow \infty $ - приближается к поперечной волне.
	 \begin{equation}
	 	\frac{H_z}{H_{\perp}} \sim \frac{\kappa}{k}, k = \frac{\omega}{c}
	 \end{equation}
	 Когда $\omega  \rightarrow \infty $, $H_{\tau} \rightarrow 0$
	 %
	 И в итоге график будет затухать.
	 %
	 Но технически такое сделать сложно, потому что $TE_{01}$ - не первая мода в таком волноводе (первая - $TE_{11}$ ). Да и исторически это уже устарело. Из-за различных шероховатостей эта мода будет рассеиваться. Чтобы не было перекачки энергии предлагалось сделать так, чтобы там распостранялась только такая волна. Каким образом реализовать: нарезать волновод на ломтики, а контакт между ними не делать, тогда мы и получим одну волну внутри него.
\end{enumerate}